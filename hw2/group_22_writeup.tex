\documentclass[letterpaper,10pt,notitlepage,fleqn]{article}

\usepackage{nopageno} %gets rid of page numbers
\usepackage{alltt}                                           
\usepackage{float}
\usepackage{color}
\usepackage{url}
\usepackage{balance}
\usepackage[TABBOTCAP, tight]{subfigure}
\usepackage{enumitem}
\usepackage{pstricks, pst-node}
\usepackage{geometry}
\geometry{textheight=9in, textwidth=6.5in} %sets 1" margins 
\newcommand{\cred}[1]{{\color{red}#1}} %command to change font to red
\newcommand{\cblue}[1]{{\color{blue}#1}} % ...blue
\usepackage{hyperref}
\usepackage{textcomp}
\usepackage{listings}

\def\name{Group 22}

\parindent = 0.0 in
\parskip = 0.2 in

\title{Project 2 Write Up}
\author{Group 22}

\begin{document}
\maketitle
\hrule

\section*{Our Solution}
For this project, we started by grepping through the Linux Kernel v3.0.4 for the noop implementation that was mentioned in the assignment. We knew it was in the ../block folder and found the implementation to be in the noop-iosched.c file. Most of the functions in this C file could be kept as they were. The only changes we really made were to the data structure of a SSTF and the dispatch. Noop operates by taking the next request from a queue and then dispatching it, regardless of position. SSTF, or Shortest Seek Time First, behaves different since it selects an I/O request that is closer to its current position. As a result, we need to add two new variables to the SSTF data structure so that the function has positions to compare the values with. 

Next, we needed to heavily modify the dispatch function so that it works as a SSTF scheduler. First, we have to make sure that the queue is not empty so that we may have values to compare. The current position is then checked against the head and the direction that it is going. If it is going forward and the request is ahead of the list head, then only those values will be compared until the smallest absolute distance between the two are found. If you are going backwards, then you are only considering and comparing against the values behind the head. After the smallest absolute values is found, that I/O process is dispatched. Afterwards, the head is changed either to the front or bottom of the sector that it made the dispatch on for future dispatches. 

\end{verbatim}
\end{document}
