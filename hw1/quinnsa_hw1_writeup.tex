\documentclass[letterpaper,10pt,notitlepage,fleqn]{article}

\usepackage{nopageno} %gets rid of page numbers
\usepackage{alltt}                                           
\usepackage{float}
\usepackage{color}
\usepackage{url}
\usepackage{balance}
\usepackage[TABBOTCAP, tight]{subfigure}
\usepackage{enumitem}
\usepackage{pstricks, pst-node}
\usepackage{geometry}
\geometry{textheight=9in, textwidth=6.5in} %sets 1" margins 
\newcommand{\cred}[1]{{\color{red}#1}} %command to change font to red
\newcommand{\cblue}[1]{{\color{blue}#1}} % ...blue
\usepackage{hyperref}
\usepackage{textcomp}
\usepackage{listings}

\def\name{Sam Quinn}

\parindent = 0.0 in
\parskip = 0.2 in

\title{Project 1 Write Up}
\author{Sam Quinn}

\begin{document}
\maketitle
\hrule

\section*{What do you think the main point of this assignment is?}
I think the main purpose of this assignment was to investigate the Linux kernel first hand. This was a 
good way to become familiar with the Linux kernel build process.
\section*{How did you personally approach the problem? Design decisions, algorithm, etc.}
After learning that RR and FIFO scheduling algorithms come standard in the Linux kernel 
from researching we decided to compare the version of the Linux kernel that we had to the stock Linux kernel. 
I greped through the stock Linux kernel source code and found that the scheduling algorithms are 
contained with in the sched.c and sched_rt.c I ran a diff between the two versions and found that in 
the project 1's source code was missing a few functions that are necessary for the RR and FIFO scheduling 
algorithms. After finding that the sched files differed I created a patch file with the RR and FIFO implementation.

\section*{How did you ensure your solution was correct? Testing details, for instance.}
To verify my RR and FIFO scheduling algorithms were successfully implemented I recompiled the Linux kernel 
after applying the patch to the sched files. Once I had the kernel built I installed the image with a unique name 
and label and booted into it. Once I was running my system on the new 3.0.4 kernel I then could issue the system call
sched_setscheduler. 
\section*{What did you learn?}
I like this project because I learned how to build the Linux kernel from scratch and how to use patch files. While 
I have built other large projects from source before I have never built such a crucial part of a system before like 
the kernel. It was also very cool to use the patch function and create patch with diff, a tool I use all of the time.
\end{document}
