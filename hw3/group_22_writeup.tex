\documentclass[letterpaper,10pt,notitlepage,fleqn]{article}

\usepackage{nopageno} %gets rid of page numbers
\usepackage{alltt}                                           
\usepackage{float}
\usepackage{color}
\usepackage{url}
\usepackage{balance}
\usepackage[TABBOTCAP, tight]{subfigure}
\usepackage{enumitem}
\usepackage{pstricks, pst-node}
\usepackage{geometry}
\geometry{textheight=9in, textwidth=6.5in} %sets 1" margins 
\newcommand{\cred}[1]{{\color{red}#1}} %command to change font to red
\newcommand{\cblue}[1]{{\color{blue}#1}} % ...blue
\usepackage{hyperref}
\usepackage{textcomp}
\usepackage{listings}

\def\name{Group 22}

\parindent = 0.0 in
\parskip = 0.2 in

\title{Project 2 Write Up}
\author{Group 22}

\begin{document}
\maketitle
\hrule

\section*{Our Plan}
    To create an encrypted \textit{RAM Disk} we decided to first get a unencrypted 
    version working before we begin to work with the \textit{Linux Crypto API}. 
    After reading over the assignment details and requirements we took a look at 
    the ``Linux Device Drivers'' example we discovered that finding a working example 
    of a \textit{RAM Disk} would not be too hard to find. After hearing about how 
    undocumented the \textit{Linux Crypto API} is in class we will begin our research 
    early. 

\section*{Our Solution}
    We took advantage of reusing the code from the ``Linux Device Drivers'' book that 
    was recommended for us to get used to programming Linux Drivers. We wanted to get 
    a working \textit{RAM Disk} working before we implemented our \textit{cipher}. 
    We added the sbd.c text that we grabbed from the book and after a few compile warnings 
    we found a blog by ``Pat Patterson'' that said he had fixed the warnings and errors 
    from the sbd.c file from the ``Linux Device Drivers''. Once we compiled the kernel with 
    the ramdisk included it was we needed to mount it. Using the command \textit{fdisk -l} 
    we discovered that our new \textit{RAM Disk} was mapped to \textit{/dev/sbd0} 
    since the \textit{RAM Disk} is just a unformated block of memory we decided to make a 
    ext2 filesystem and locally mounted it to a folder. After the Disk was mounted we copied 
    a few files to it and made sure it worked OK before we continued.\\
    Finding examples and any documentation for that mater was very difficult for the 
    \textit{Linux Crypto API} but after a bit of searching we eventually came to the conclusion 
    that you need to
    \begin{enumerate}
        \item Allocate crypto API
        \item Set cipher key
        \item Encrypt/Decrypt one byte at a time
        \item Free the cipher
    \end{enumerate}
    We attempted to use the ``Blowfish'' cipher algorithm instead of the standard ``AES'' 
    to try something different from the few examples we came across. It was hard to use the 
    \textit{Linux Crypto API} macros since they were fairly ambiguous. After a while of fiddling 
    we got our Kernel to compile and boot. Once the system started we formated the drive and mounted 
    it like we did before.\\
    To verify that our solution was correct we took advantage of the printk function and printed the 
    RAW data to the console before and after it was \textit{encrypted}. It was very easy to examine 
    the differences from before and after the \textit{encryption} took place. Even though we were printing 
    unsigned chars directly from the buffer and already couldn't read the data it was clear that something 
    changed when the \textit{encrypted} data was printed.

\section*{Work Log}

\begin{center}
    \begin{tabular}{| p{3cm} | l | l | p{5cm} |}
        \hline
        Date & Author & Commit & Summary \\ \hline
        Tue Oct 14 15:00 & Sam Quinn & 9402a8ddf13ad69b80eb9bf42c294822aff87b2f & Added the Linux folder for homework \#2.
        \\ \hline
        Fri Oct 17 13:39 & Bob & 8daeb619e3ebf201b079b3643e83c70c119f8de0 & Coppied the Noop I/) scheduler for a template in creating the new SSTF scheduler.
        \\ \hline
        Wed Oct 22 18:42 & Bob & 6fbc561c60d26ea444b063b912ac01068f5fca44 & Implemented sstf\_dispatch
        \\ \hline
        Sat Oct 25 15:22 & Bob & 3c4249ea31657ed2d5eadf9fa9e8ac52923a38d7 & Fixed case logic in SSTF\_dispatch
        \\ \hline
        Sat Oct 25 15:55 & Bob & b57104fba0b40f41f79c19d08310b543f8b7f6b9 & Added a hw2 Writeup doc.
        \\ \hline
        Sat Oct 25 21:36 & Bob & c918023c4d2f92f087e1cd721bde987b6291fe5a & Added Sam's project 2 writeup file.
        \\ \hline
        Sun Oct 26 21:57 & Bob & 4737476b348f7fb7acc3b47daa5328d41d393117 & Update sstf-iosched.c fully working.
        \\ \hline
        Mon Oct 27 02:38 & lawrencechau & 0fad02a3acef842d3d4b7ec6d4b1ebca2781cf29 & Update group\_22\_writeup.tex, Wrote the group writeup
        \\ \hline
        Mon Oct 27 11:53 & Bob & 50ea3799a520d50c9ee0f32ff3bf0f7b3b56a180 & Added the patch file with our SSTF I/O Scheduler.
        \\ \hline
        Mon Oct 27 11:55 & Bob & e06f0b76c25ebca39a0051d0132ba7c8ad74bf54 & Added the Linux Kernel source as a git ignored file.
        \\ \hline
        Mon Nov 10 21:12 & Bob & af5c9f7eab1da3f867639174c34c4a1b97f2a394 & Finnished the group witeup.
        \\ \hline
    \end{tabular}
\end{center}
\end{document}




commit
Author: bob <bob@CS444.(none)>
Date:   Mon Nov 10 21:12:10 2014 -0800

    

commit cd1a064e295c87c3cafeee62842ca5b379735bc3
Author: bob <bob@CS444.(none)>
Date:   Mon Nov 10 20:53:40 2014 -0800

    Added the group Writeup tex and makefile

commit e744aa955ac823fb16abccbf23752374e396500f
Author: bob <bob@CS444.(none)>
Date:   Mon Nov 10 18:27:58 2014 -0800

    Added the kernel patch file for the encrypted ram disk.

commit b3e07347b213607769bfcf75739449ecc6a4be62
Author: bob <bob@CS444.(none)>
Date:   Mon Nov 10 18:13:49 2014 -0800

    Added the tar ball for Sam's writeup.

commit e9e015e98d3b8e04c2a8762b4af20029ce72245a
Author: bob <bob@CS444.(none)>
Date:   Mon Nov 10 18:09:46 2014 -0800

    Added a pdf version of Sam;s Writeup.

commit 689a1ac1ff20bd4d32c1f318984277c05cb8cf85
Author: Sam Quinn <Sam>
Date:   Mon Nov 10 18:08:54 2014 -0800

    Updated Sam's writeup with finished version.

commit 93a01769fcd7d5cb670419a7c68baa4a91dda6a6
Author: bob <bob@CS444.(none)>
Date:   Mon Nov 10 11:01:17 2014 -0800

    Added the outline for Sam's Writeup.

commit 3a9436d55efb23125f914d22181830a5e5f2d85d
Author: bob <bob@CS444.(none)>
Date:   Mon Nov 10 10:56:23 2014 -0800

    Updated the .gitignore file.

commit 3d5d384d6174a0667bb7c0b07e0bd21702b8151f
Author: bob <bob@CS444.(none)>
Date:   Mon Nov 10 10:53:52 2014 -0800

    Added original unecrypted ramdisk code from LLD3

commit 3899e2b81125105ff1e1c6d3c08b6e7e4e8b152f
Author: bob <bob@CS444.(none)>
Date:   Mon Nov 10 10:49:47 2014 -0800

    Added a mounting shell script to mount the encrypted ramdisk in one command.

commit b907ba3f829fae8178c7a05eb62e925eed699f56
Author: bob <bob@CS444.(none)>
Date:   Mon Nov 10 10:45:32 2014 -0800

    Added a working version of the encrypted Ramdisk.

commit 3f23e27044ff6f1d896075a36974c258c17a84f4
Author: bob <bob@CS444.(none)>
Date:   Wed Nov 5 19:25:59 2014 -0800

    Finnished Makefuile and tar.bz2 for Concurrency 3

commit ce1c354c64c832cd137e17617b258c1c116034e0
Author: bob <bob@CS444.(none)>
Date:   Wed Nov 5 19:21:12 2014 -0800

    Finnised concurrency 3.

commit 8feccdab21ba8750a0c5e9efe128aa44aa4fd3c4
Merge: 2cebcc9 0955dd5
Author: bob <bob@CS444.(none)>
Date:   Sun Nov 2 12:20:50 2014 -0800

    Merge branch 'master' of github.com:quinnsam/cs444

commit 2cebcc999819ba670771c438eb14f813e99fc989
Author: bob <bob@CS444.(none)>
Date:   Sun Nov 2 12:20:35 2014 -0800

    Added hw3 folder and some documentation.

